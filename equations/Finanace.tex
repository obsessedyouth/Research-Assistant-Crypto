\documentclass{article}

\usepackage[english]{babel}
\usepackage[utf8]{inputenc}
\usepackage{amsmath,amssymb}
\usepackage{parskip}
\usepackage{graphicx}

% Margins
\usepackage[top=2.5cm, left=3cm, right=3cm, bottom=4.0cm]{geometry}
% Colour table cells
\usepackage[table]{xcolor}

% Get larger line spacing in table
\newcommand{\tablespace}{\\[1.25mm]}
\newcommand\Tstrut{\rule{0pt}{2.6ex}}         % = `top' strut
\newcommand\tstrut{\rule{0pt}{2.0ex}}         % = `top' strut
\newcommand\Bstrut{\rule[-0.9ex]{0pt}{0pt}}   % = `bottom' strut

%%%%%%%%%%%%%%%%%
%     Title     %
%%%%%%%%%%%%%%%%%
\title{Research Assistant Crypto Response}
\author{Haile Lagi}
\date{September 26, 2020}

\begin{document}
\maketitle

%%%%%%%%%%%%%
% Finance   %
%%%%%%%%%%%%%

\section{Finance}
Yara Inc is listed on the NYSE with a stock price of 40 - the company is not known
to pay dividends. We need to price a call option with a strike of \$45 maturing in
4 months.
The continuously-compounded risk-free rate is 3\%/year, the mean return on the
stock is 7\%/year, and the standard deviation of the stock return is 40\%/year.
What is the Black-Scholes call price?

Given that the black-scholes model is
\begin{align}
    \label{eq:black_scholes} % black-scholes model
    N(x) = \frac{1}{\sqrt(2\pi)}\int_{-\infty }^{x}e^{\frac{-x^2}{2}}dz
\end{align}

Let's dig in!!!!!! :)

So the solutions are $y=some value$ and $x = some value$.

\end{document}
