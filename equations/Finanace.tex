\documentclass{article}

\usepackage[english]{babel}
\usepackage[utf8]{inputenc}
\usepackage{amsmath,amssymb}
\usepackage{parskip}
\usepackage{graphicx}

% Margins
\usepackage[top=2.5cm, left=3cm, right=3cm, bottom=4.0cm]{geometry}
% Colour table cells
\usepackage[table]{xcolor}

% Get larger line spacing in table
\newcommand{\tablespace}{\\[1.25mm]}
\newcommand\Tstrut{\rule{0pt}{2.6ex}}         % = `top' strut
\newcommand\tstrut{\rule{0pt}{2.0ex}}         % = `top' strut
\newcommand\Bstrut{\rule[-0.9ex]{0pt}{0pt}}   % = `bottom' strut

%%%%%%%%%%%%%%%%%
%     Title     %
%%%%%%%%%%%%%%%%%
\title{Research Assistant Crypto Response}
\author{Haile Lagi}
\date{September 26, 2020}

\begin{document}
\maketitle

%%%%%%%%%%%%%
% Finance   %
%%%%%%%%%%%%%

\section{Finance}
Yara Inc is listed on the NYSE with a stock price of 40 - the company is not known
to pay dividends. We need to price a call option with a strike of \$45 maturing in
4 months.
The continuously-compounded risk-free rate is 3\%/year, the mean return on the
stock is 7\%/year, and the standard deviation of the stock return is 40\%/year.
What is the Black-Scholes call price?

Parameters given are (assuming the stock doesn't pay dividend it is assumed an American call option is equivalent to an European call option):

Converting all parameters w.r.t. to 1 year:

Given that:
\begin{itemize}
\item Black-Scholes call price = $(C)$
\item  stock price$(S)$ = \$40
\item call option strike$(X)$ = \$45
\item  maturity$(T)$ = 4 months / 12 = 1/3
\item continuously-compounded risk-free rate$(r_{f})$ = 3/100 = 0.03
\item stock mean return$(\mu)$ = 7/100 = 0.07
\item standard deviation($\sigma$) = 40/100 = 0.4
\end{itemize}

the linear black-scholes general form for an European style call option is:
\begin{align}
    \label{eq:call_option}
    C(S, T) = SN(x_{1}) - BN(x_{2})
\end{align}

Where:
\begin{align}
   Bond price (B) = Xe^-r_{f} * T
\end{align}

and $x_{1}$ is given by:

\begin{align}
  x_{1} = \frac{log(S/B)}{\sigma \sqrt{T}} + \frac{1}{2}\sigma \sqrt{T}
\end{align}


and to find $x_{2}$ from the value of $x_{1}$:

\begin{align}
  x_{2} = x_{1} - \sigma * \sqrt{T}
\end{align}

To find the value of $N(x)$ which is the cumulative normal distribution we use the integral form:

\begin{align}
    \label{eq:gaussian}
    N(x) = \frac{1}{\sqrt(2\pi)}\int_{-\infty }^{x}e^{\frac{-u^2}{2}}du
\end{align}

First we find $B = Xe^-r_{f} * T = 40e^-0.03 * \frac{1}{3} $
$B = 44.552243$

Using this we find $x_{1}$

\begin{align}
  x_{1} = \frac{log(40/44.552243)}{0.4 \sqrt{1/3}} + \frac{1}{2} * 0.4 * \sqrt{1/3}
\end{align}

\begin{align}
  = \frac{-0.10778304645914898}{0.2309401076758503} + (0.5 * 0.4 * 0.2309401076758503)
\end{align}


\begin{align}
  x_{1}  \approx
  -0.3512439362402078046572242 \approx -0.3512439
\end{align}

\begin{align}
  x_{2}  = -0.3512439362402078046572242 - 0.4 * \sqrt{1/3} \approx
  -0.5821840439160581104608837 \approx -0.5821840
\end{align}

Now we must find the Gaussian probability distribution functions of $N(x_{1})$ and $N(x_{2})$

\begin{align}
    \label{eq:gaussian_x_one}
    N(x_{1}) = \frac{1}{\sqrt(2\pi)}\int_{-\infty }^{-0.3512439}e^{\frac{-u^2}{2}}du
\end{align}

\begin{align}
    \label{eq:gaussian_x_two}
    N(x_{1}) = \frac{1}{\sqrt(2\pi)}\int_{-\infty }^{-0.5821840}e^{\frac{-u^2}{2}}du
\end{align}

Sadly the cumulative difference function of a normal distribution does not have
a closed form and CANNOT BE SOLVED analytically. The accuracy of the values produced by each estimation methods varies in accuracy. I'll be using a $Z-table$ or $Standard Normal Talbe$

\begin{align}
  N(x_{1}) = 0.3527026 N(x_{2}) = 0.2802213
\end{align}

Then finally substitution into the general form:
\begin{align}
    C(S, T) = 40 * 0.3627026 - 44.55224 * 0.2802213
\end{align}

The Black-Scholes call price is \approx \$2.02361738929
\end{document}
