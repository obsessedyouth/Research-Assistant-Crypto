\documentclass{article}

\usepackage[english]{babel}
\usepackage[utf8]{inputenc}
\usepackage{amsmath,amssymb}
\usepackage{parskip}
\usepackage{graphicx}

% Margins
\usepackage[top=2.5cm, left=3cm, right=3cm, bottom=4.0cm]{geometry}
% Colour table cells
\usepackage[table]{xcolor}

% Get larger line spacing in table
\newcommand{\tablespace}{\\[1.25mm]}
\newcommand\Tstrut{\rule{0pt}{2.6ex}}         % = `top' strut
\newcommand\tstrut{\rule{0pt}{2.0ex}}         % = `top' strut
\newcommand\Bstrut{\rule[-0.9ex]{0pt}{0pt}}   % = `bottom' strut

%%%%%%%%%%%%%%%%%
%     Title     %
%%%%%%%%%%%%%%%%%
\title{Research Assistant Crypto Response}
\author{Haile Lagi}
\date{September 25, 2020}

\begin{document}
\maketitle

%%%%%%%%%%%%%%%%%
% Math problem   %
%%%%%%%%%%%%%%%%%
\section{Math}
 Over all real numbers, find the minimum value of a positive real number, $y$ such that
\begin{align}
    \label{eq:obj_fn} % Equation label; can be used for referencing
    y = \sqrt{(x+6)^2 + 25)} + \sqrt{(x-6)^2 + 121} \,.
\end{align}

To find a stationary point(in this case mininum) for \ref{eq:obj_fn} the procedure is as follows:
\begin{enumerate}
\item Given y = f(x) find f'(x)
\item Let $\frac{dy}{dx}=0$ and solve for the x value(s).
\item find the corresponding y value(s).
\item Determine the nature of the equation using:
\begin{itemize}
\item Second derivative and substitute x if x is positive, negative or zero
\end{itemize}
\begin{itemize}
\item Alternatively, find gradient before and after the stationary + to -, - to - or + to +
\end{itemize}
\end{enumerate}

First we simplify \ref{eq:obj_fn} by expanding the polynomials of the form
$(x + c)^n$ and we have

\begin{align}
    \label{eq: linear}
    y = \sqrt{(x+6)(x+6) + 25)} + \sqrt{(x-6)(x-6) + 121}
    = \sqrt{x^2+12x+61} + \sqrt{x^2 - 12x + 157} \,.
\end{align}

To find f'(x) using chain rule, in an equation of the form
$\frac{dy}{dx}=u^n$ where u is function of x and n is some exponent this holds
$\frac{dy}{dx}=\frac{dy}{du} * \frac{du}{dx}$. Divide equation \ref{eq: linear}
into two parts $y = y_{1} + y_{2}$ where

\begin{align}
    \label{eq: linear_one}
    y_{1} = \sqrt{x^2+12x+61}  \,.
\end{align}

\begin{align}
\label{eq: linear_two}
    y_{2} = \sqrt{x^2 - 12x + 157} \,.
\end{align}

Applying the chain rule in \ref{eq: linear_one} and \ref{eq: linear_two}
\begin{align}
    \label{eq: y_one}
    \frac{dy_{1}}{dx} = u^\frac{1}{2} \,.
\end{align}

where $u_{1} = x^2 + 12x + 61$ and $\frac{du_{1}}{dx} = 2x + 12$

\begin{align}
    \label{eq: y_two}
    \frac{dy_{2}}{dx} = u^\frac{1}{2} \,.
\end{align}

where $u_{2} = x^2 - 12x - 12x + 157$ and $\frac{du_{2}}{dx} = 2x - 12$

Combining \ref{eq: y_one} and \ref{eq: y_two} we have \ref{eq: diffed}

\begin{align}
    \label{eq: diffed}
\frac{dy}{dx} = \frac{2x + 12}{2 * \sqrt(x^2 + 12x + 61)} + \frac{2x - 12}{2 * \sqrt(x^2 - 12x + 157)} \,.
\end{align}

By simple factorization of 2 and Let $\frac{dy}{dx}=0$

\begin{align}
    \label{eq: irration}
 0 = \frac{x + 6}{(x^2 + 12x + 61)^\frac{1}{2}} + \frac{x - 6}{(x^2 - 12x + 157)^\frac{1}{2}} \,.
\end{align}

At this point it becomes impossible to solve for x because the values of x are complex(or imaginary) so, the root(s) of x are \textbf{unknown}.

\begin{align}
    \label{eq: solve_y}
 y = (x^2 + 12x + 61)^\frac{1}{2} + (x^2 - 12x + 157)^\frac{1}{2} \,.
\end{align}

The minimum value of a real x is assumed to be  0

Substituting $x = 0$ in \ref{eq: solve_y} we have the minimum value of y as:
\begin{align}
    \label{eq: final}
 y = \sqrt{61} + \sqrt{157} \,.
\end{align}

Finally we have $y_{min} = (0, \sqrt{61} + \sqrt{157})$

\end{document}
