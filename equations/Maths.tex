\documentclass{article}

\usepackage[english]{babel}
\usepackage[utf8]{inputenc}
\usepackage{amsmath,amssymb}
\usepackage{parskip}
\usepackage{graphicx}

% Margins
\usepackage[top=2.5cm, left=3cm, right=3cm, bottom=4.0cm]{geometry}
% Colour table cells
\usepackage[table]{xcolor}

% Get larger line spacing in table
\newcommand{\tablespace}{\\[1.25mm]}
\newcommand\Tstrut{\rule{0pt}{2.6ex}}         % = `top' strut
\newcommand\tstrut{\rule{0pt}{2.0ex}}         % = `top' strut
\newcommand\Bstrut{\rule[-0.9ex]{0pt}{0pt}}   % = `bottom' strut

%%%%%%%%%%%%%%%%%
%     Title     %
%%%%%%%%%%%%%%%%%
\title{Research Assistant Crypto Response}
\author{Haile Lagi}
\date{September 22, 2020}

\begin{document}
\maketitle

%%%%%%%%%%%%%%%%%
% Math problem   %
%%%%%%%%%%%%%%%%%
\section{Math}
 Over all real numbers, find the minimum value of a positive real number, $y$ such that
\begin{align}
    \label{eq:obj_fn} % Equation label; can be used for referencing
    y = \sqrt{(x+6)^2 + 25)} + \sqrt{(x-6)^2 + 121} \,.
\end{align}
This $some equation$ and apply it to Equation~\ref{eq:obj_fn}. This gives us

\begin{align}
\end{align}

So the solutions are $y=some value$ and $x = some value$.

\end{document}
